\documentclass{article}
\usepackage[T1]{fontenc}
\usepackage{multirow}
\usepackage{geometry}
\usepackage{amsmath, graphicx, geometry, siunitx, pgfplots, physics}
\usepackage{booktabs}
\usepackage[czech]{babel}
\usepackage[utf8]{inputenc}
\usepackage{hyperref}
\usepackage{indentfirst}
\usepackage{float}
\usepackage{appendix}
\usepackage{bm}
\usepackage{makecell}


\geometry{a4paper}

\sisetup{detect-all, mode=text}

\setlength{\heavyrulewidth}{1.8pt} 
\setlength{\lightrulewidth}{1.0pt}
\setlength{\parindent}{1cm}
\setlength{\parskip}{0.25em}  

\newcommand{\header}[1]{%
	\rule{0pt}{2\baselineskip}%
	\textbf{\makecell[c]{#1}}%
}

\newcommand{\headermath}[1]{%
	\rule{0pt}{2\baselineskip}%
	\boldmath%
	\let\oldSI\si\renewcommand{\si}[1]{\textbf{\oldSI{##1}}}%
	$\displaystyle #1$%
}

\begin{document}
	\begin{titlepage}
		\begin{center}
			\vspace*{1cm}
			Vysoká škola chemicko-technologická, Praha\\
			Fakulta chemického inženýrství\\
			Ústav fyzikální chemie (403)\\
			
			\vfill
			
			\textbf{\huge Isingův model ve 3D}
			
			\vspace{0.5cm}
			P02-ISING
			
			\vspace{1.5cm}
			
			\textbf{Jakub Vencel}
			
			Semestrální práce \\
			Počítačová chemie (B403011)\\
			
			\vspace{0.8cm}
			
			\includegraphics[width=0.2\textwidth]{logoVSCHT_ikona.png}
			
			
			Praha 2025 \\
			vedoucí práce: prof. RNDr. Jiří Kolafa, CSc.\\
			
		\end{center}
	\end{titlepage}
	
	\section{Úvod}
	
	\subsection{}
	
\end{document}