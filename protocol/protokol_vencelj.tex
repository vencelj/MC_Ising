\documentclass{article} % doc type
\usepackage[czech]{babel} % labnguage support
\usepackage[T1]{fontenc} % font
\usepackage{multirow} % multirow
\usepackage{geometry} % paper geometry
\usepackage{amsmath} % math for LaTeX
\usepackage{amssymb} % mathbb
\usepackage{graphicx} % pictures, graphical env
% https://pgfplots.sourceforge.net/pgfplots.pdf
\usepackage{pgfplots} % graph
% https://mirrors.ibiblio.org/CTAN/macros/latex/contrib/physics/physics.pdf
\usepackage{physics} % special math symbols
% https://texdoc.org/serve/siunitx/0
\usepackage{siunitx} % tables, quantities
\usepackage{booktabs} % 
\usepackage{hyperref} % functional links
\usepackage{indentfirst} % indent the first row in paragraph
\usepackage{float} % override graphical placement
\usepackage{appendix} % appendicies
\usepackage{bm} % bold math
\usepackage{makecell} % split text within cell
\usepackage{url} % url
%https://mirrors.ibiblio.org/CTAN/macros/latex/contrib/biblatex-contrib/biblatex-iso690/biblatex-iso690.pdf
\usepackage[style=iso-authoryear, backend=biber]{biblatex} % ISO 690


\addbibresource{protocol-ref.bib}

\geometry{a4paper}

% setup SIunitX env
\sisetup{
	output-decimal-marker = {,},
	retain-explicit-plus = true,
	uncertainty-mode = separate,
	list-final-separator = {~a~},
	list-pair-separator  = {~a~},
	range-phrase={~do~},
	per-mode = power,
	inter-unit-product = \ensuremath{\cdot},
	exponent-product = {\, \ensuremath{\cdot} \,},
	uncertainty-separator = {\, \ensuremath{\pm} \,},
	tight-spacing = false,
}

% tables
\setlength{\heavyrulewidth}{1.8pt} % table separator
\setlength{\lightrulewidth}{1.2pt} % header separator

% text header
\newcommand{\header}[1]{% 
	\rule{0pt}{2\baselineskip}%
	\textbf{\makecell[c]{#1}}%
}

% math header
\newcommand{\headermath}[2]{%
	\rule{0pt}{2\baselineskip}%
	\boldmath%
	$\displaystyle \frac{#1}{\unit[unit-font-command = \mathbf]{#2}}$%
}

% renew SiunitX v. 3.X.X
\AtBeginDocument{\RenewCommandCopy\qty\SI}
\ExplSyntaxOn
% turn off wanring about SIunitX & physics conflict
\msg_redirect_name:nnn { siunitx } { physics-pkg } { none } 
\ExplSyntaxOff
\begin{document}
	\begin{titlepage}
		\begin{center}
			\vspace*{1cm}
			Vysoká škola chemicko-technologická, Praha\\
			Fakulta chemického inženýrství\\
			Ústav fyzikální chemie (403)\\
			
			\vfill
			
			\textbf{\huge Isingův model ve 3D}
			
			\vspace{0.5cm}
			P02-ISING
			
			\vspace{1.5cm}
			
			\textbf{Jakub Vencel}
			
			Semestrální práce \\
			Počítačová chemie (B403011)\\
			
			\vspace{0.8cm}
			
			\includegraphics[width=0.2\textwidth]{logoVSCHT_ikona.png}
			
			
			Praha 2026 \\
			vedoucí práce: prof. RNDr. Jiří Kolafa, CSc.\\
			
		\end{center}
	\end{titlepage}
	
	\section{Úvod}
	\subsection{Isingův model}
	Předpokládejme kubickou mřížku $\mathbb{Z}^3$ o velikosti $L \cross L \cross L$. V takové mřížce se nachází $N = L^3$ prvků, které mají spin $\sigma = \left\{-1; +1\right\}$ \parencite{Viswanathan2022}. Pro tuto mřížku můžeme definovat Hemiltonián $H(\sigma)$ (platí pro případ, kdy je nulové vnější magnetické pole)
	\begin{equation} \label{hamil_eq}
		H(\sigma) = - J \sum_{<ij>}\left(\sigma_i\sigma_j\right),
	\end{equation}
	kde $J$ je interakční energie, která nabývá hodnot $J > 0$ pro feromagnety a $J < 0$ pro antiferomagnety.
	
	Pro feromagnetické látky musí platit, že konfigurace spinů je taková, aby vznikl nenulový magnetický moment $M$, který lze vypočítat jako součet všech spinů v mřížce
	\begin{equation} \label{mag_eq}
		M(\sigma) = \sum_{i=1}^{N}\sigma_i.
	\end{equation}
	Obdobně lze vypočítat energii spinové konfigurace mřížky $E$ \parencite{FitzpatrickIsing}
	\begin{align} \label{energy_eq}
		e_i = - \frac{J}{2} \sum_{<ij>}\left(\sigma_i\sigma_j\right), \\
		E(\sigma) = \sum_{i=1}^{N}e_i.
	\end{align}
	
	Spiny v mřížce mají uspořádání dané Boltzmannovou distribucí \parencite{CambridgeCATAM11pt2}
	\begin{equation} \label{boltz_dist}
		p(\boldsymbol{\sigma}|T) = \exp(\frac{E(\sigma)}{\mathbf{k_B}T}),
	\end{equation}
	kde $\mathbf{k_B}$ je Boltzmannova konstanta a $T$ je teplota. Do kritické teploty $T_c$ (někdy zvané i jako Curieova teplota) jsou spiny v dostatečném uspořádání, aby magnetický moment $M(\sigma)$ měl nenulovou hodnotu. V $T_c$ dochází k fázové přeměně druhého druhu a v teplotách $T>T_c$ se moment ztrácí \parencite{Hasenbusch_1998}. Pro 3D Isingův model byla inverzní kritická teplota numericky vypočtena s výsledkem $\beta_c = \qty{0.2216595 \pm 0.0000026}{\per \kelvin}$ \parencite{PhysRevB.44.5081}.
	
	\subsection{Monte Carlo}
	Jelikož neexistuje analytické řešení Isingova modelu pro 3D mřížku, přichází na pomoc numerická simulace Monte Carlo (MC simulace).
	\subsubsection{Ukázka důkazu MC}
	\subsubsection{Okrajové podmínky}
	\subsubsection{Typy algoritmů}
	
	\section{Program}
	\subsection{Simulace}
	\subsection{Uživatelské rozhraní}
	
	\section{Výsledky}
	\subsection{Konstantní teplota}
	\subsection{Teplotní cykly}
	\subsection{Antiferomagnet}
	\subsection{Hystereze – vliv počátečních podmínek na průběh simulace}
	
	\newpage
	\tableofcontents
	
	\newpage
	\printbibliography[title={Reference}]
	
	\newpage
	\begin{appendices}
	\end{appendices}
	
\end{document}