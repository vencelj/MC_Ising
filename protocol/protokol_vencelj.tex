\documentclass{article} % doc type
\usepackage[czech]{babel} % labnguage support
\usepackage[T1]{fontenc} % font
\usepackage{multirow} % multirow
\usepackage{geometry} % paper geometry
\usepackage{amsmath} % math for LaTeX
\usepackage{graphicx} % pictures, graphical env
\usepackage{pgfplots} % graph
\usepackage{physics} % derivative
\usepackage{siunitx} % tabůes, quantities
\usepackage{booktabs} % 
\usepackage{hyperref} % functional links
\usepackage{indentfirst} % indent the first row in paragraph
\usepackage{float} % override graphical placement
\usepackage{appendix} % appendicies
\usepackage{bm} % bold math
\usepackage{makecell} % split text with cell
\usepackage{url} % url


\geometry{a4paper}

% setup SIunitX env
\sisetup{
	output-decimal-marker = {,},
	retain-explicit-plus = true,
	uncertainty-mode = separate,
	list-final-separator = {~a~},
	list-pair-separator  = {~a~},
	range-phrase={~do~},
	per-mode = power,
	inter-unit-product = \ensuremath{\cdot},
	exponent-product = {\, \ensuremath{\cdot} \,},
	uncertainty-separator = {\, \ensuremath{\pm} \,},
	tight-spacing = false,
}

% tables
\setlength{\heavyrulewidth}{1.8pt} % table separator
\setlength{\lightrulewidth}{1.2pt} % header separator


\newcommand{\header}[1]{% 
	\rule{0pt}{2\baselineskip}%
	\textbf{\makecell[c]{#1}}%
}

\newcommand{\headermath}[2]{%
	\rule{0pt}{2\baselineskip}%
	\boldmath%
	$\displaystyle \frac{#1}{\unit[unit-font-command = \mathbf]{#2}}$%
}

% renew SiunitX v. 3.X.X
\AtBeginDocument{\RenewCommandCopy\qty\SI}
\ExplSyntaxOn
% turn off wanring about SIunitX & physics conflict
\msg_redirect_name:nnn { siunitx } { physics-pkg } { none } 
\ExplSyntaxOff
\begin{document}
	\begin{titlepage}
		\begin{center}
			\vspace*{1cm}
			Vysoká škola chemicko-technologická, Praha\\
			Fakulta chemického inženýrství\\
			Ústav fyzikální chemie (403)\\
			
			\vfill
			
			\textbf{\huge Isingův model ve 3D}
			
			\vspace{0.5cm}
			P02-ISING
			
			\vspace{1.5cm}
			
			\textbf{Jakub Vencel}
			
			Semestrální práce \\
			Počítačová chemie (B403011)\\
			
			\vspace{0.8cm}
			
			\includegraphics[width=0.2\textwidth]{logoVSCHT_ikona.png}
			
			
			Praha 2025 \\
			vedoucí práce: prof. RNDr. Jiří Kolafa, CSc.\\
			
		\end{center}
	\end{titlepage}
	
	\section{Úvod}
	$$\qty{5 \pm 0.1}{\raiseto{0.5}\meter}$$
	\qtylist{5; 2; 4; 5}{\meter \second}
	$$\qtyproduct{10 x 20 x 30}{\meter}$$
	\qtyrange{5}{30}{\meter}
	$$\qty{317 \pm 0.01e-7}{\m\per\s}$$
	$$\ang{10;20;30}$$
	$$\complexqty{1.5 + 2.3j}{\ohm}$$
	$$\qty{60}{px \per \second}$$
	$$\qty{60}{px \, \second^{-1}}$$
	$$\qty{5}{\meter\textsubscript{peak}}$$
	$$\qty{230}{\volt\of{in}}$$
	\begin{table}[H]
		\centering
		\caption{Training SIunitX table}
		\label{My first SIunitX table}
		\sisetup{
			table-alignment-mode = format,
			table-auto-round = true,
		}
		\begin{tabular}{
				c
				c 
				S[table-format = 4.1(1)] 
				c
			}
			\toprule
			{\headermath{\nu}{\per\cm}} & {\header{Intenzita}} & {\header{Vibrace}} & {\header{Skupina}} \\
			\midrule
			3102--3001 & m   & 1452 \pm 0.2 & Ar \\ 
			1601         & v   & 750.1(5)       & Ar \\
			\bottomrule
		\end{tabular}
	\end{table}
	
\end{document}